\begin{center}
    \begin{adjustbox}{width=\textwidth}
        \begin{tabularx}{\textwidth}{|X|X|}
        \hline
        \multirow{2}{*}{\textbf{Transaction-1 ($T_{1}$)}} & \multirow{2}{*}{\textbf{Transaction-2 ($T_{2}$)}} \\
        & \\ \hline
        & \textsc{Read($Q_{1}$)} \\
        & \textsc{$Q := Q - q_{2}$} \\
        & \textsc{Write($Q$)} \\
        \textsc{Read($Q$)} & \\
        \textsc{$Q := Q + q_{1}$} & \\
        \textsc{Write($Q$)} & \\
        & \textsc{Read($B_{2}$)} \\
        & \textsc{Check($B_{2}$)} \\
        & \textsc{Write($B_{2}'$)} \\
        & \textsc{Write(Order $O_{2}$)} \\
        \hline
        \end{tabularx}
    \end{adjustbox}
\end{center}
\vspace*{10pt}
This schedule is conflict serializable as we can repeatedly move the three statements from $T_{1}$ upwards, such that
both the transactions get executed serially ($T_{2}$ after $T_{1}$).