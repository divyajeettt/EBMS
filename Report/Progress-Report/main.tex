\documentclass[12pt]{report}
\usepackage[utf8]{inputenc}
\usepackage{adjustbox}
\usepackage{geometry}
\usepackage{hyperref}
\usepackage{multicol}
\usepackage{multirow}
\usepackage{graphicx}
\usepackage{listings}
\usepackage{tabularx}
\usepackage{xcolor}
\usepackage{array}
\usepackage{color}

\geometry{bottom=30mm, top=30mm, left=30mm, right=30mm}

\newcommand{\deadline}[3]{
    \hspace{0pt}
    \vfill
    \begin{center}
        \Huge \textbf{Deadline #1} \\
        \vspace*{5pt}
        \Large \textit{#2} \\
        \vspace*{25pt}
        \large \textbf{#3}
    \end{center}
    \vfill
    \pagebreak
}

\definecolor{dkgreen}{rgb}{0, 0.6, 0.0}
\definecolor{gray}{rgb}{0.5, 0.5, 0.5}
\definecolor{mauve}{rgb}{0.58, 0.0, 0.82}

\lstset{frame=tb,
    language=SQL,
    aboveskip=3mm,
    belowskip=3mm,
    showstringspaces=false,
    columns=flexible,
    basicstyle={\footnotesize\ttfamily},
    numbers=none,
    numberstyle=\tiny\color{gray},
    keywordstyle=\color{blue},
    commentstyle=\color{dkgreen},
    stringstyle=\color{mauve},
    breaklines=true,
    breakatwhitespace=true,
    tabsize=4
}

\title{
    \textbf{\Huge ElectroBase Management System} \\
    \vspace*{10pt}
    \large{\textit{EBMS - an Online Electronics Retail Store}} \\
    \vspace*{20pt}
    Built as a course project for \\
    CSE202: Fundamentals of Database Management Systems \\
    \vspace*{25pt}
    \textbf{\Large Project Progress Report and Documentation}
    \vfill
}

\author{
    \href{mailto:divyajeet21529@iiitd.ac.in}{Divyajeet Singh (2021529)} \\
    \vspace*{20pt}
    \href{mailto:mehar21541@iiitd.ac.in}{Mehar Khurana (2021541)}
}

\date{
    \vfill
    \textbf{\today}
    \vspace*{15pt}
    \begin{center}
        \includegraphics[scale=0.9]{Logo-1}
    \end{center}
}

\graphicspath{{./../Assets/}}

\begin{document}
    \maketitle

    \input{Deadlines/deadline-01.tex}
    \deadline{2}{Designing the ER Model and converting it to a Relational Model}{February 3, 2023}

\section*{\Huge Entity-Relationship Model}
\vspace*{10pt}
Entity-Relationship (ER) Models are used to plan how different entities in a project interact with each other. \\
\newline
Our ER Model captures the nature of the relationships and entities planned to be used in the project.
The ER Model is designed in accordance with the assumptions and constraints as mentioned in the document above.
Hence, we plan to build our system on the basis of the following Entity-Relationship Model:
\vspace*{10pt}
\begin{center}
    \hspace*{-37pt}
    \includegraphics[scale=0.6]{ER-Model}
\end{center}

\subsection*{Ternary Relationships}
The following ternary relationships have been identified:
\begin{enumerate}
    \item \textbf{Customer - Product - Product Review:}
    A customer can review multiple products, and a product can be reviewed by multiple customers.
    A customer can give at most one review per product.
    This ternary relationship will be decomposed into the following binary relationships at the time of implentation:
    \begin{enumerate}
        \item \textbf{Customer - Product:}
        (Many-to-Many) To keep track of which customers have purchased which products.
        \item \textbf{Product - Product Review:}
        (One-to-Many) To keep track of all reviews given to a product.
        \item \textbf{Customer - Product Review:}
        (One-to-Many) To keep track of all reviews given by a customer.
        This relationship may not be needed and may be removed in the future.
    \end{enumerate}
    \item \textbf{Delivery Agent - Customer - Delivery Agent Review:}
    A customer can review multiple delivery agents, and a delivery agent can be reviewed by multiple customers.
    A customer can give at most one review per delivery agent.
    This ternary relationship will be decomposed into the following binary relationships at the time of implentation:
    \begin{enumerate}
        \item \textbf{Customer - Delivery Agent:}
        (Many-to-Many) To keep track of which customers have received orders from which delivery agents.
        \item \textbf{Delivery Agent - Delivery Agent Review:}
        (One-to-Many) To keep track of all reviews given to a delivery agent.
        \item \textbf{Customer - Delivery Agent Review:}
        (One-to-Many) To keep track of all reviews given by a customer.
        This relationship may not be needed and may be removed in the future.
    \end{enumerate}
\end{enumerate}

\section*{\Huge Relational Model}
\vspace*{10pt}
Relationship Models are used to represent how data will be stored in the database, along with the attributes of each entity and relationship.
The Relational Model is designed in accordance with the assumptions and constraints as mentioned in the document above. \\
\newline
\textbf{Note:} The arrows represent that a field is \textit{derived} from another.
For example, \texttt{productID} in \texttt{description} will contain values of \texttt{productID} from table \texttt{product}.
\pagebreak
\vspace*{35pt}
\begin{center}
    \hspace*{-40pt}
    \includegraphics[scale=0.8]{Relational-Model}
\end{center}

\pagebreak
    \deadline{3}{Implementing the Database Schema and Integrity Constraints, and Populating simulated data satisfying them}{February 10, 2023}

\section*{\Huge Iterations on the Database Schema}
\subsection*{Tables added or renamed:}
\begin{enumerate}
    \item
    Since \textbf{address} is a composite attribute, we store it in a separate table since we would rarely need to search for the address.
    The search/join/union query efficiency would get affected if we store all the address data in the same stakeholder table.
    \item
    The \textbf{consists\_of} relationship table has been renamed to \textbf{order\_product} to make it more intuitive.
    \item
    Entity table \textbf{description} has been renamed to \textbf{product\_description}, since \newline \texttt{description} is a reserved keyword in MySQL.
    Its field \textbf{description} has also been renamed to \textbf{content} for the same reason.
    \item
    Entity table \textbf{order} has also been renamed to \textbf{orders} for the same reason.
\end{enumerate}

\subsection*{Columns/Fields added, removed, or renamed:}
\begin{enumerate}
    \item
    All fields called \textbf{hashed\_password} have been renamed to \textbf{pwd} for simplicity.
    \item
    Added field \textbf{email} to \textbf{customer}, \textbf{supplier}, and \textbf{delivery\_agent} tables.
    \item
    Added field \textbf{quantity} to table \textbf{product}.
    \item
    Removed fields \textbf{reviewID} from the \textbf{product\_review} and \textbf{da\_review} tables,
    so as to decompose the ternary relationships \textbf{delivered} and \textbf{purchased}.
    \item
    Added \textbf{order\_date} and \textbf{delivery\_date} to the \textbf{order} table.
\end{enumerate}

\section*{\Huge Handling Different Attributes}
\subsection*{Composite Attributes:}
\begin{enumerate}
    \item
    Addresses of all stakeholders are stored in a single table with a unique \textbf{addressID} assigned to each address.
    This \textbf{addressID} is then stored along with other stakeholder data in their tables.
    \item
    Added sub-attribute \textbf{country} to the \textbf{address} composite attribute.
    \item
    The \textbf{name} composite attribute is stored in the same table, since we might need to search for the names of the stakeholders.
    Each sub-attribute is kept as a column (\textbf{first\_name}, \textbf{middle\_initial}, \textbf{last\_name}) in the table.
\end{enumerate}

\subsection*{Multi-Valued Attributes:}
\begin{enumerate}
    \item
    Since \textbf{phone\_number} is a multivalued attribute, we store it in a separate table, with the \textbf{phoneID} attribute
    from this table being stored in the \textbf{delivery\_agent} and \textbf{customer} tables.
    This table does not have a primary key since each \textbf{phoneID} associates to one or more \textbf{num}'s (phone numbers).
\end{enumerate}

\subsection*{Derived Attributes:}
\begin{enumerate}
    \item
    Changed \textbf{ETA} and \textbf{delivered} (earlier \textbf{status}) to derived attributes.
    \textbf{ETA} will be calculated as \textbf{order\_date} + 15 days when required.
    \item
    The attribute \textbf{delivered} will be \texttt|True| if \textbf{delivery\_date} is not null, else \texttt|False|.
\end{enumerate}
Since these constraints/relations will be implemented outside the database, and hence these fields are not present in the tables.

\section*{Assumptions:}
\begin{enumerate}
    \item
    The field \textbf{delivery\_date} of \textbf{orders} is allowed to be \texttt{NULL}.
    This is because we do not know the \textbf{delivery\_date} when the order is placed, and it is only updated when the order is delivered.
    We derive the attribute \textbf{delivered} from \textbf{delivery\_date} to check if the order has been delivered or not.
    \item
    Since \textbf{phone numbers} are implemented as multi-valued attributes, we assume that phone numbers are not unique to a customer.
    This means that multiple users may have the same phone number.
    \item
    All primary keys are defined with the \texttt{AUTO\_INCREMENT} constraints so that we do not need to insert ID values ourselves,
    and duplicity errors on primary keys are avoided.
    \item
    \textbf{Product descriptions} and \textbf{review contents} are being stored using \texttt{TEXT} datatype in MySQL, which is not stored
    in the server memory and does not hamper query times.
    \item
    All attributes that can not have null values have been specified as \texttt{NOT NULL}. Attributes like \textbf{last\_name}, \textbf{middle\_initial},
    \textbf{content} in the \textbf{review} tables, and \textbf{delivery\_date} in the \textbf{orders} table can have null values.
    \item
    \textbf{Availability} of a delivery agent has been given a \texttt{DEFAULT} value of true, ie, when a delivery agent is added to the database,
    they are available to deliver an order by default.
    \item
    \textbf{Balance} in a customer wallet, on account creation, has been given a \texttt{DEFAULT} value of 0.
    \item
    \textbf{pwd}'s in all tables are stored as SHA1 hashes.
    \item
    Some other integrity constrains have also been added, for example, \textbf{rating} must be between 0 and 5, and \textbf{quantity}
    and \textbf{price} must be positive.
    \item
    Based on a rough idea of the types of queries we plan to use, some indices have also been added on the fields of the tables.
    These may be updated in the future.
\end{enumerate}

\section*{\Huge Data Generation \& Population}
\subsection*{Data Generation}
Most of the simulated data for the stakeholder and main entity tables was generated using \href{https://www.mockaroo.com}{\color{blue}\underline{https://www.mockaroo.com}}. \\
\newline
We utilised the (Ruby) code functionality to implement viable constraints on the data while data generation.
The data for each table was downloaded as a CSV-file.
We then used python scripts to generate the MySQL insertion queries and to populate relations.
Data for some tables (like \textbf{cart} and \textbf{orders}) was mainly done through python scripts so as to make sure the existential constraints were not violated.

\subsection*{Data Population}
All the tables of the database were pre-populated with data with integrity-constrains maintained to start querying.
The database was populated with the following number of rows of data:
\begin{multicols}{2}
    \begin{itemize}
        \item \texttt{address}: 400 rows
        \item \texttt{phone\_number}: 400 rows
        \item \texttt{admin}: 2 rows
        \item \texttt{supplier}: 200 rows
        \item \texttt{customer}: 200 rows
        \item \texttt{delivery\_agent}: 200 rows
        \item \texttt{product}: 200 rows
        \item \texttt{orders}: 1000 rows
        \item \texttt{wallet}: 200 rows
        \item \texttt{product\_review}: 200 rows
        \item \texttt{da\_review}: 200 rows
        \item \texttt{cart}: 1073 rows
        \item \texttt{order\_product}: 5552 rows
    \end{itemize}
\end{multicols}

\pagebreak
    \deadline{4}{SQL Queries and Relational Algebraic Operations}{February 17, 2023}

\section*{\Huge SQL Queries}
We attempted to implement the most relevant queries for the application using the database.
These queries are utilized by different stakeholders to perform their tasks.
The following is a list of SQL Queries and their use cases:

\begin{enumerate}
    \item The following queries are used while placing an order.

\begin{lstlisting}
-- Find out an available delivery agent
SELECT daID FROM delivery_agent
WHERE availability = 1 ORDER BY daID ASC LIMIT 1;

-- Add the order to the table orders and assign the order to the selected delivery agent
INSERT INTO orders (customerID, daID, order_date)
VALUES (50, (SELECT daID FROM delivery_agent
WHERE availability = 1 ORDER BY daID ASC LIMIT 1), '2021-04-01');

-- Add the ordered products in the table order_product
INSERT INTO order_product (orderID, productID, quantity)
SELECT MAX(o.orderID), c.productID, c.quantity
FROM orders o INNER JOIN cart c ON o.customerID = c.customerID
WHERE o.customerID = 50
GROUP BY c.productID;

-- Reduce the quantity of products ordered from table product
UPDATE product p INNER JOIN cart c ON p.productID = c.productID
SET p.quantity = p.quantity - c.quantity
WHERE c.customerID = 50;

-- Delete the products from the cart
DELETE FROM cart WHERE customerID = 50;

-- Update the wallet balance of the customer
UPDATE wallet w
INNER JOIN orders o ON w.customerID = o.customerID
SET w.balance = w.balance - (
    SELECT SUM(p.price * op.quantity)
    FROM product p, order_product op
    WHERE p.productID = op.productID AND op.orderID = o.orderID
);

-- Update the availability of the delivery agent
UPDATE delivery_agent
SET availability = 0
WHERE daID = (
    SELECT daID FROM orders WHERE customerID = 50 AND orderID = (
        SELECT MAX(orderID) FROM orders WHERE customerID = 50
    )
);
\end{lstlisting}
    \item This query is used to get a list of all suppliers who have all their products with average rating above 3.

\begin{lstlisting}
SELECT
    s.supplierID,
    CONCAT(s.first_name, ' ', s.middle_initial, ' ', s.last_name) AS name
FROM supplier s
WHERE (
    SELECT AVG(pr.rating) FROM product_review pr, product p
    WHERE p.productID = pr.productID AND p.supplierID = s.supplierID
    GROUP BY p.supplierID
) > 3;
\end{lstlisting}
    \item \textbf{Feature:} These queries are used to get an overview of statistics for the admin. \\
\textbf{Location:} In function \texttt{admin()} in \texttt{front-end/\_\_init\_\_.py}

\begin{lstlisting}
SELECT COUNT(*) AS customer_count FROM customer;
SELECT COUNT(*) AS supplier_count FROM supplier;
SELECT COUNT(*) AS da_count FROM delivery_agent;
SELECT COUNT(*) AS order_count FROM orders;
SELECT COUNT(*) AS product_count FROM order_product;
\end{lstlisting}
    \item This query displays all the products that have average rating above or equal to 3.5 ordered by average rating.

\begin{lstlisting}
SELECT p.productID, p.name, AVG(pr.rating) AS avg_rating
FROM product_review pr, product p
WHERE p.productID = pr.productID
GROUP BY p.productID
HAVING avg_rating >= 3.5
ORDER BY avg_rating DESC;
\end{lstlisting}
    \item \textbf{Feature:} This query is used to get the number of orders by status. \\
\textbf{Location:} In function \texttt{search()} in \texttt{front-end/\_\_init\_\_.py}

\begin{lstlisting}
SELECT (delivery_date IS NULL) as status, COUNT(*) AS n
FROM orders GROUP BY status;
\end{lstlisting}
    \item This query shows the undelivered orders of a delivery agent (where delivery date is NULL).

\begin{lstlisting}
SELECT
    orderID,
    customerID,
    daID,
    order_date,
    DATE_FORMAT(ADDDATE(order_date, INTERVAL 15 DAY), '%Y-%m-%d') AS ETA
FROM orders
WHERE daID = 1 AND delivery_date IS NULL;
\end{lstlisting}
    \item \textbf{Feature:} This query is used to display the top-10 best selling products. \\
\textbf{Location:} In function \texttt{search()} in \texttt{front-end/\_\_init\_\_.py}

\begin{lstlisting}
SELECT p.productID, p.name, p.price, SUM(op.quantity) AS units_sold
FROM order_product op, product p
WHERE p.productID = op.productID
GROUP BY p.productID
ORDER BY units_sold DESC
LIMIT 10;
\end{lstlisting}
    \item This query shows the top 15 customers who have spent the most money on their orders.

\begin{lstlisting}
SELECT
    customer.customerID,
    COUNT(orders.orderID) AS total_orders,
    SUM(product.price * order_product.quantity) AS total_spent
FROM customer
INNER JOIN orders ON customer.customerID = orders.customerID
INNER JOIN order_product ON orders.orderID = order_product.orderID
INNER JOIN product ON order_product.productID = product.productID
GROUP BY customer.customerID
ORDER BY total_spent DESC
LIMIT 15;
\end{lstlisting}
    \item \textbf{Feature:} This query is used to display inactive suppliers. \\
\textbf{Location:} In function \texttt{search()} in \texttt{front-end/\_\_init\_\_.py}

\begin{lstlisting}
SELECT
    s.supplierID,
    CONCAT(s.first_name, ' ', s.middle_initial, ' ', s.last_name) AS name,
    email
FROM supplier s
WHERE NOT EXISTS (SELECT * FROM product p WHERE p.supplierID = s.supplierID)
LIMIT 10;
\end{lstlisting}
    \item \textbf{Feature:} This query is used to display the top-10 highest rated suppliers. \\
\textbf{Location:} In function \texttt{search()} in \texttt{front-end/\_\_init\_\_.py}

\begin{lstlisting}
SELECT
    s.supplierID,
    CONCAT(s.first_name, ' ', s.middle_initial, ' ', s.last_name) AS name,
    email,
    AVG(pr.rating) AS avg_rating
FROM supplier s, product_review pr, product prod
WHERE (
    SELECT AVG(pr.rating) FROM product_review pr, product p
    WHERE p.productID = pr.productID AND p.supplierID = s.supplierID
    GROUP BY p.supplierID
) > 3
AND s.supplierID = prod.supplierID AND pr.productID = prod.productID
GROUP BY s.supplierID
ORDER BY avg_rating DESC
LIMIT 10;
\end{lstlisting}
    \item This query is used to show all underlivered orders with their ETA for a customer.

\begin{lstlisting}
SELECT orderID, order_date,
DATE_FORMAT(ADDDATE(order_date, INTERVAL 15 DAY), "%Y-%m-%d") AS ETA
FROM orders
INNER JOIN customer ON orders.customerID = customer.customerID
WHERE customer.customerID = 1 AND orders.delivery_date IS NULL
ORDER BY order_date;
\end{lstlisting}
    \item This query find out the total revenue and total quantity sold per product for a supplier (sales statistics).

\begin{lstlisting}
SELECT
    product.name,
    SUM(order_product.quantity) AS total_quantity_sold,
    SUM(order_product.quantity * product.price) AS total_revenue
FROM product
INNER JOIN order_product ON product.productID = order_product.productID
INNER JOIN orders ON order_product.orderID = orders.orderID
WHERE product.supplierID = 1
GROUP BY product.name;
\end{lstlisting}
    \item This query is used to search through the product catalogue for names of the products using pattern matching.

\begin{lstlisting}
SELECT name, AVG(rating)
FROM product
JOIN product_review ON product.productID = product_review.productID
WHERE name LIKE 'LED %'
GROUP BY name;
\end{lstlisting}
    \item This query is used to add a product to a customer's cart.

\begin{lstlisting}
INSERT INTO cart (customerID, productID, quantity) VALUES (50, 1, 1);
\end{lstlisting}
    \item This query adds a new phone number into the table for a customer.

\begin{lstlisting}
INSERT INTO phone_number
VALUES ((SELECT phoneID FROM customer WHERE customerID = 50), '1234567890');
\end{lstlisting}
    \item This query adds more quantity of products for an existing product.

\begin{lstlisting}
UPDATE product SET quantity = quantity + 100 WHERE productID = 1;
\end{lstlisting}
    \item This query updates the address of a customer.

\begin{lstlisting}
UPDATE address SET
    apt_number = '100',
    street_name = 'Thomspon St.',
    city = 'Albany',
    state = 'New York',
    zip = '12207',
    country = 'United States'
WHERE addressID = (SELECT addressID FROM customer WHERE customerID = 1);
\end{lstlisting}
    \item This query deletes a product from a customer's cart.

\begin{lstlisting}
DELETE FROM cart WHERE customerID = 99 AND productID = 14;
\end{lstlisting}
\end{enumerate}

\pagebreak
    \deadline{5}{Embedded SQL Queries, OLAP Queries, and Triggers}{March 26, 2023}

\section*{\Huge Frontend Development}
\vspace*{10pt}

Since this deadline now requires embedding SQL queries in a host programming language, the frontend development has also been started.
As decided previously, the host langauge is Python-3, and the frontend is being developed using the Flask Web-Development Micro Framework. \\
The raw frontend is in the first stages of development and is hosted on a local server.
This document contains references and bibliographies for all open-source code available on the internet for tasks like placing sortable tables on a page,
CSS animations, and JavaScript functions for form validation.

\section*{\Huge Embedded SQL Queries}
\vspace*{10pt}

As of now, the only SQL queries which have been embedded in the Python code are those for which the frontend interaces have been developed.
In the future, these will be refactored into small \texttt{.sql} files and imported into the Python code for security\footnote{
    The embedded SQL Queries are currently not protected against SQL Injection Attacks.
} and readability purposes.
\vspace*{10pt} \\
The following is a (non-exhaustive) list of already embedded SQL queries:

\begin{enumerate}
    \item \textbf{Feature:} This query is used to display the entire product catalogue. \\
\textbf{Location:} In function \texttt{search()} in \texttt{front-end/\_\_init\_\_.py}

\begin{lstlisting}
SELECT p.name, p.price, AVG(pr.rating) AS avg_rating, p.quantity
FROM product p, product_review pr
WHERE p.productID = pr.productID
GROUP BY p.productID
ORDER BY p.name ASC;
\end{lstlisting}

A variation of the same query (which uses pattern matching in SQL) is also used to search through the product catalogue.
\footnote{
    \ In this variant of the query, \texttt{search} is a string which the user wishes to search for.
    The string is substituted into the query using an \texttt{f-string} in Python.
}

\begin{lstlisting}
SELECT p.name, p.price, AVG(pr.rating) AS avg_rating, p.quantity
FROM product p, product_review pr
WHERE p.productID = pr.productID AND p.name LIKE '%{search}%'
GROUP BY p.productID
ORDER BY p.name ASC;
\end{lstlisting}

This variant can also be found in the function \texttt{search()} in \texttt{front-end/\_\_init\_\_.py}. \\
    \item \textbf{Feature:} This query is used to get the login credentials of a customer. \\
\textbf{Location:} In function \texttt{login()} in \texttt{front-end/\_\_init\_\_.py}

\begin{lstlisting}
SELECT customerID, first_name, email, pwd FROM customer WHERE email = %s;
\end{lstlisting}

A variation of the same query is also used for supplier login.
\footnote{
    \ \texttt{\%s} denotes a placeholder in Python, which will be substituted by the value entered by the user in the login form.
}

\begin{lstlisting}
SELECT supplierID, first_name, email, pwd FROM supplier WHERE email = %s;
\end{lstlisting}

Another variant will also be added for delivery agent login.
This variant can also be found in the function \texttt{login()} in \texttt{front-end/\_\_init\_\_.py}. \\
    \item \textbf{Feature:} These queries are used to get an overview of statistics for the admin. \\
\textbf{Location:} In function \texttt{admin()} in \texttt{front-end/\_\_init\_\_.py}

\begin{lstlisting}
SELECT COUNT(*) AS customer_count FROM customer;
SELECT COUNT(*) AS supplier_count FROM supplier;
SELECT COUNT(*) AS da_count FROM delivery_agent;
SELECT COUNT(*) AS order_count FROM orders;
SELECT COUNT(*) AS product_count FROM order_product;
\end{lstlisting}
    \item \textbf{Feature:} This query is used to retrieve the demographics of supplier activity. \\
\textbf{Location:} In function \texttt{admin\_stats(page)} in \texttt{front-end/\_\_init\_\_.py}

\textbf{Usage:}
When the HTML page is rendered, the rolled-up data is separated by regional demographics (country and state).
The data contains the number of suppliers that are selling products on EBMS grouped by country and state.
Empty values in the field \texttt{state} indicate that the data is an aggregate of country-wise data.
The data record where country is empty gives the global aggregated data.
The average earning per order along with total earning is also shown.

\vspace*{30pt}

\begin{lstlisting}
SELECT
    country, state,
    COUNT(DISTINCT supplier.supplierID) AS supplier_count,
    AVG(product.price * order_product.quantity) AS avg_earned,
    SUM(product.price * order_product.quantity) AS total_earned
FROM supplier
JOIN orders ON supplier.supplierID IN (
    SELECT supplierID FROM product WHERE productID IN (
        SELECT productID FROM order_product WHERE orderID = orders.orderID
    )
)
JOIN order_product ON orders.orderID = order_product.orderID
JOIN product ON order_product.productID = product.productID
JOIN address ON supplier.addressID = address.addressID
GROUP BY country, state WITH ROLLUP
ORDER BY country ASC, total_earned DESC
\end{lstlisting}
    \item \textbf{Feature:} This query is used to get the number of orders by status. \\
\textbf{Location:} In function \texttt{search()} in \texttt{front-end/\_\_init\_\_.py}

\begin{lstlisting}
SELECT (delivery_date IS NULL) as status, COUNT(*) AS n
FROM orders GROUP BY status;
\end{lstlisting}
    \item \textbf{Feature:} This query is used to display the customers having the least orders. \\
\textbf{Location:} In function \texttt{search()} in \texttt{front-end/\_\_init\_\_.py}

\begin{lstlisting}
SELECT
    customer.customerID,
    CONCAT(customer.first_name, ' ', customer.middle_initial, ' ', customer.last_name) AS name,
    COUNT(orders.orderID) AS total_orders,
    SUM(product.price * order_product.quantity) AS total_spent
FROM customer
INNER JOIN orders ON customer.customerID = orders.customerID
INNER JOIN order_product ON orders.orderID = order_product.orderID
INNER JOIN product ON order_product.productID = product.productID
GROUP BY customer.customerID
HAVING total_orders <= 10
ORDER BY total_orders ASC
LIMIT 10;
\end{lstlisting}
    \item \textbf{Feature:} This query is used to display the top-10 best selling products. \\
\textbf{Location:} In function \texttt{search()} in \texttt{front-end/\_\_init\_\_.py}

\begin{lstlisting}
SELECT p.productID, p.name, p.price, SUM(op.quantity) AS units_sold
FROM order_product op, product p
WHERE p.productID = op.productID
GROUP BY p.productID
ORDER BY units_sold DESC
LIMIT 10;
\end{lstlisting}
    \item \textbf{Feature:} This query is used to display the top-10 highest rated products. \\
\textbf{Location:} In function \texttt{search()} in \texttt{front-end/\_\_init\_\_.py}

\begin{lstlisting}
SELECT p.productID, p.name, p.price, AVG(pr.rating) AS avg_rating
FROM product_review pr, product p
WHERE p.productID = pr.productID
GROUP BY p.productID
HAVING avg_rating >= 3.5
ORDER BY avg_rating DESC
LIMIT 10;
\end{lstlisting}
    \item \textbf{Feature:} This query is used to display inactive suppliers. \\
\textbf{Location:} In function \texttt{search()} in \texttt{front-end/\_\_init\_\_.py}

\begin{lstlisting}
SELECT
    s.supplierID,
    CONCAT(s.first_name, ' ', s.middle_initial, ' ', s.last_name) AS name,
    email
FROM supplier s
WHERE NOT EXISTS (SELECT * FROM product p WHERE p.supplierID = s.supplierID)
LIMIT 10;
\end{lstlisting}
    \item \textbf{Feature:} This query is used to display the top-10 highest rated suppliers. \\
\textbf{Location:} In function \texttt{search()} in \texttt{front-end/\_\_init\_\_.py}

\begin{lstlisting}
SELECT
    s.supplierID,
    CONCAT(s.first_name, ' ', s.middle_initial, ' ', s.last_name) AS name,
    email,
    AVG(pr.rating) AS avg_rating
FROM supplier s, product_review pr, product prod
WHERE (
    SELECT AVG(pr.rating) FROM product_review pr, product p
    WHERE p.productID = pr.productID AND p.supplierID = s.supplierID
    GROUP BY p.supplierID
) > 3
AND s.supplierID = prod.supplierID AND pr.productID = prod.productID
GROUP BY s.supplierID
ORDER BY avg_rating DESC
LIMIT 10;
\end{lstlisting}
    \item \textbf{Feature:} This query is used to display the top-10 highest rated delivery agents. \\
\textbf{Location:} In function \texttt{search()} in \texttt{front-end/\_\_init\_\_.py}

\begin{lstlisting}
SELECT
    da.daID,
    CONCAT(da.first_name, ' ', da.middle_initial, ' ', da.last_name) AS name,
    email,
    AVG(dr.rating) AS avg_rating
FROM da_review dr, delivery_agent da
WHERE da.daID = dr.daID
GROUP BY da.daID
ORDER BY avg_rating DESC
LIMIT 10;
\end{lstlisting}
    \item \textbf{Feature:} This query is used to display the top-10 highest rated delivery agents. \\
\textbf{Location:} In function \texttt{search()} in \texttt{front-end/\_\_init\_\_.py}

\begin{lstlisting}
SELECT
    da.daID,
    CONCAT(da.first_name, ' ', da.middle_initial, ' ', da.last_name) AS name,
    email,
    COUNT(o.orderID) AS total_orders
FROM delivery_agent da
LEFT JOIN orders o ON da.daID = o.daID
GROUP BY da.daID
ORDER BY total_orders DESC
LIMIT 10;
\end{lstlisting}
\end{enumerate}

\section*{\Huge OLAP SQL Queries}
\vspace*{10pt}

OLAP queries are (mainly) utilised by the Admin in our use case to perform data analytics.
The OLAP data is separated from the rolled-up form before rendering the HTML page. These queries are also embedded in Python.
\vspace*{10pt} \\
The following is a (non-exhaustive) list of already implemented OLAP SQL queries:

\begin{enumerate}
    \item The following queries are used while placing an order.

\begin{lstlisting}
-- Find out an available delivery agent
SELECT daID FROM delivery_agent
WHERE availability = 1 ORDER BY daID ASC LIMIT 1;

-- Add the order to the table orders and assign the order to the selected delivery agent
INSERT INTO orders (customerID, daID, order_date)
VALUES (50, (SELECT daID FROM delivery_agent
WHERE availability = 1 ORDER BY daID ASC LIMIT 1), '2021-04-01');

-- Add the ordered products in the table order_product
INSERT INTO order_product (orderID, productID, quantity)
SELECT MAX(o.orderID), c.productID, c.quantity
FROM orders o INNER JOIN cart c ON o.customerID = c.customerID
WHERE o.customerID = 50
GROUP BY c.productID;

-- Reduce the quantity of products ordered from table product
UPDATE product p INNER JOIN cart c ON p.productID = c.productID
SET p.quantity = p.quantity - c.quantity
WHERE c.customerID = 50;

-- Delete the products from the cart
DELETE FROM cart WHERE customerID = 50;

-- Update the wallet balance of the customer
UPDATE wallet w
INNER JOIN orders o ON w.customerID = o.customerID
SET w.balance = w.balance - (
    SELECT SUM(p.price * op.quantity)
    FROM product p, order_product op
    WHERE p.productID = op.productID AND op.orderID = o.orderID
);

-- Update the availability of the delivery agent
UPDATE delivery_agent
SET availability = 0
WHERE daID = (
    SELECT daID FROM orders WHERE customerID = 50 AND orderID = (
        SELECT MAX(orderID) FROM orders WHERE customerID = 50
    )
);
\end{lstlisting}
    \item \textbf{Feature:} This query is used to find out the country-wise trends in the number of orders and revenue. \\
\textbf{Location:} In function \texttt{admin\_stats(page)} in \texttt{front-end/\_\_init\_\_.py}

\textbf{Usage:}
When the HTML page is rendered, the rolled-up data is separated.
Due to an overwhelmingly large number of states, cities, etc., the data is grouped by country only.
The query can be extended to rollup with states, cities, etc. by adding them to the \texttt{SELECT} and \texttt{GROUP BY} clause.
The data record where country is empty gives the total global revenue and order count.

\vspace*{10pt}

\begin{lstlisting}
SELECT
    a.country,
    COUNT(DISTINCT o.orderID) AS order_count,
    SUM(op.quantity * p.price) AS revenue
FROM orders o
JOIN order_product op ON o.orderID = op.orderID
JOIN product p ON op.productID = p.productID
JOIN customer c ON o.customerID = c.customerID
JOIN address a ON c.addressID = a.addressID
GROUP BY a.country WITH ROLLUP
ORDER BY revenue DESC;
\end{lstlisting}
    \item \textbf{Feature:} This query is used to retrieve the demographics of customer activity. \\
\textbf{Location:} In function \texttt{admin\_stats(page)} in \texttt{front-end/\_\_init\_\_.py}

\textbf{Usage:}
When the HTML page is rendered, the rolled-up data is separated by regional demographics (country and state).
The data contains the number of customers that are registered on EBMS grouped by country and state.
Empty values in the field \texttt{state} indicate that the data is an aggregate of country-wise data.
The data record where country is empty gives the global aggregated data.
The average spending per order along with total spending is also shown.

\vspace*{30pt}

\begin{lstlisting}
SELECT
    country, state,
    COUNT(DISTINCT customer.customerID) AS customer_count,
    AVG(product.price * order_product.quantity) AS avg_spent,
    SUM(product.price * order_product.quantity) AS total_spent
FROM customer
JOIN orders ON customer.customerID = orders.customerID
JOIN order_product ON orders.orderID = order_product.orderID
JOIN product ON order_product.productID = product.productID
JOIN address ON customer.addressID = address.addressID
GROUP BY country, state WITH ROLLUP
ORDER BY country ASC, total_spent DESC
\end{lstlisting}

    \item \textbf{Feature:} This query is used to retrieve the demographics of supplier activity. \\
\textbf{Location:} In function \texttt{admin\_stats(page)} in \texttt{front-end/\_\_init\_\_.py}

\textbf{Usage:}
When the HTML page is rendered, the rolled-up data is separated by regional demographics (country and state).
The data contains the number of suppliers that are selling products on EBMS grouped by country and state.
Empty values in the field \texttt{state} indicate that the data is an aggregate of country-wise data.
The data record where country is empty gives the global aggregated data.
The average earning per order along with total earning is also shown.

\vspace*{30pt}

\begin{lstlisting}
SELECT
    country, state,
    COUNT(DISTINCT supplier.supplierID) AS supplier_count,
    AVG(product.price * order_product.quantity) AS avg_earned,
    SUM(product.price * order_product.quantity) AS total_earned
FROM supplier
JOIN orders ON supplier.supplierID IN (
    SELECT supplierID FROM product WHERE productID IN (
        SELECT productID FROM order_product WHERE orderID = orders.orderID
    )
)
JOIN order_product ON orders.orderID = order_product.orderID
JOIN product ON order_product.productID = product.productID
JOIN address ON supplier.addressID = address.addressID
GROUP BY country, state WITH ROLLUP
ORDER BY country ASC, total_earned DESC
\end{lstlisting}
    \item \textbf{Feature:} This query is used to find out a supplier's statistics by date and region. \\
\textbf{Location:} This query has not been added yet, as it is a supplier-specific query and supplier-pages are not yet implemented.

\textbf{Usage:}
A supplier's sales grouped by a particular month/year and country are returned by this query.
We can separate this rolled up data into monthly, yearly, and country-wise trends.
\footnote{
    Here, \texttt{s\_id} is the supplier ID of the currently logged in supplier.
    The string is substituted into the query using an \texttt{f-string} in Python.
}

\vspace*{10pt}

\begin{lstlisting}
SELECT
    YEAR(order_date) AS year,
    MONTH(order_date) AS month,
    country,
    COUNT(DISTINCT o.orderID) AS total_quantity,
    SUM(op.quantity * p.price) AS total_sales
FROM orders o
JOIN order_product op ON o.orderID = op.orderID
JOIN product p ON op.productID = p.productID
JOIN customer c ON o.customerID = c.customerID
JOIN address a ON c.addressID = a.addressID
JOIN supplier s ON p.supplierID = s.supplierID
WHERE s.supplierID = {s_id}
GROUP BY year, month, a.country WITH ROLLUP;
\end{lstlisting}
\end{enumerate}

\section*{\Huge Triggers}

The following triggers are identified according to the use cases and implemented in the database:

\begin{enumerate}
    \item \textbf{Trigger Name:} \texttt{create\_wallet} \\
This trigger is used to create a wallet for a customer when the record for the customer is created.

\begin{lstlisting}
DROP TRIGGER IF EXISTS create_wallet;
DELIMITER $$
CREATE TRIGGER create_wallet AFTER INSERT ON customer
FOR EACH ROW
BEGIN
    IF NOT EXISTS (SELECT * FROM wallet WHERE customerID = NEW.customerID) THEN
        INSERT INTO wallet (customerID, balance) VALUES (NEW.customerID, 0);
    END IF;
END;
$$
DELIMITER ;
\end{lstlisting}
    \item \textbf{Trigger Name:} \texttt{da\_unavailable} \\
This trigger is used to toggle a delivery agent's availability to False when they are assigned to an order

\begin{lstlisting}
DROP TRIGGER IF EXISTS da_unavailable;
DELIMITER $$
CREATE TRIGGER da_unavailable AFTER INSERT ON orders
FOR EACH ROW
BEGIN
    UPDATE delivery_agent SET avalability = FALSE WHERE daID = NEW.daID;
END;
$$
DELIMITER ;
\end{lstlisting}
    \item \textbf{Trigger Name:} \texttt{da\_unavailable} \\
This trigger is used to toggle a delivery agent's availability to to True when they complete an order

\begin{lstlisting}
DROP TRIGGER IF EXISTS da_available;
DELIMITER $$
CREATE TRIGGER da_available AFTER UPDATE ON orders
FOR EACH ROW
BEGIN
    IF NOT EXISTS (
        SELECT * FROM orders WHERE daID = NEW.daID AND delivery_date IS NULL
    ) THEN
        UPDATE delivery_agent SET avalability = TRUE WHERE daID = NEW.daID;
    END IF;
END;
$$
DELIMITER ;
\end{lstlisting}
\end{enumerate}

\vfill \pagebreak
    \deadline{6}{Conflicting and Non-Conflicting Database Transactions}{April 24, 2023}

\section*{\Huge Concurrent Database Transactions}
\vspace*{10pt}
Assuming the application is hosted on a public server, multiple users can access and use it simultaneously.
Hence, there may arise cases where multiple users access the database (through transactions) in a way that creates conflicts and inconsistency.

\subsection*{Transaction Pair 1}
Consider the case of two customers buying the same product simultaneously.
The following table of transactions sums up the reads and writes to the database needed to complete the required actions. \\
Here, $Q$ represents the quantity (in stock) of Product 1. Similarly, $B_{1}$ and $B_{2}$ represent the balance of
customers 1 and 2, whereas their orders are represented by $O_{1}$ and $O_{2}$, respectively.

\begin{center}
    \begin{adjustbox}{width=\textwidth}
        \begin{tabularx}{\textwidth}{|X|X|}
        \hline
        \multirow{2}{*}{\textbf{Transaction-1 ($T_{1}$)}} & \multirow{2}{*}{\textbf{Transaction-2 ($T_{2}$)}} \\
        & \\ \hline
        \textsc{Read(product 1 quantity)} & \textsc{Read(product 1 quantity)} \\
        $Q := Q - q_{1}$ & $Q := Q - q_{2}$ \\
        \textsc{Write(product 1 quantity)} & \textsc{Write(product 1 quantity)} \\
        \textsc{Read(balance user 1)} & \textsc{Read(balance user 2)} \\
        \textsc{Write(new balance user 1)} & \textsc{Write(new balance user 2)} \\
        \textsc{Commit} & \textsc{Commit} \\
        \hline
        \end{tabularx}
    \end{adjustbox}
\end{center}

\vspace*{2.5pt}

\subsection*{Conflict-Serializable Schedule}
\begin{center}
    \begin{adjustbox}{width=\textwidth}
        \begin{tabularx}{\textwidth}{|X|X|}
        \hline
        \multirow{2}{*}{\textbf{Transaction-1 ($T_{1}$)}} & \multirow{2}{*}{\textbf{Transaction-2 ($T_{2}$)}} \\
        & \\ \hline
        & \textsc{Lock-X($Q$)} \\
        & \textsc{Read($Q$)} \\
        & \textsc{$Q := Q - q_{2}$} \\
        & \textsc{Write($Q$)} \\
        & \textsc{Unlock($Q$)} \\
        & \textsc{Lock-X($B_{2}$)} \\
        & \textsc{Read($B_{2}$)} \\
        & \textsc{Write($B_{2}'$)} \\
        & \textsc{Unlock($B_{2}$)} \\
        \textsc{Lock-X($Q$)} & \\
        \textsc{Read($Q$)} & \\
        \textsc{$Q := Q - q_{1}$} & \\
        \textsc{Write($Q$)} & \\
        \textsc{Unlock($Q$)} & \\
        & \textsc{Lock-X(Order $O_{2}$)} \\
        & \textsc{Write(Order $O_{2}$)} \\
        & \textsc{Unlock(Order $O_{2}$)} \\
        \textsc{Lock-X($B_{1}$)} & \\
        \textsc{Read($B_{1}$)} & \\
        \textsc{Check($B_{1}$)} & \\
        \textsc{Write($B_{1}'$)} & \\
        \textsc{Unlock($B_{1}$)} & \\
        \textsc{Lock-X(Order $O_{1}$)} & \\
        \textsc{Write(Order $O_{1}$)} & \\
        \textsc{unlock(Order $O_{1}$)} & \\
        \hline
        \end{tabularx}
    \end{adjustbox}
\end{center}
\vspace*{10pt}
We add locks before each read, and unlock after each write.
This ensures that no two transactions are accessing/altering the same data point at the same time,
which could lead to conflicts.

\subsection*{Conflict-Serializable Schedule with Locks}
\begin{center}
    \begin{adjustbox}{width=\textwidth}
        \begin{tabularx}{\textwidth}{|X|X|}
        \hline
        \multirow{2}{*}{\textbf{Transaction-1 ($T_{1}$)}} & \multirow{2}{*}{\textbf{Transaction-2 ($T_{2}$)}} \\
        & \\ \hline
        & \textsc{Lock-X($Q$)} \\
        & \textsc{Read($Q$)} \\
        & \textsc{$Q := Q - q_{2}$} \\
        & \textsc{Write($Q$)} \\
        & \textsc{Unlock($Q$)} \\
        & \textsc{Lock-X($B_{2}$)} \\
        & \textsc{Read($B_{2}$)} \\
        & \textsc{Write($B_{2}'$)} \\
        & \textsc{Unlock($B_{2}$)} \\
        \textsc{Lock-X($Q$)} & \\
        \textsc{Read($Q$)} & \\
        \textsc{$Q := Q - q_{1}$} & \\
        \textsc{Write($Q$)} & \\
        \textsc{Unlock($Q$)} & \\
        & \textsc{Lock-X(Order $O_{2}$)} \\
        & \textsc{Write(Order $O_{2}$)} \\
        & \textsc{Unlock(Order $O_{2}$)} \\
        \textsc{Lock-X($B_{1}$)} & \\
        \textsc{Read($B_{1}$)} & \\
        \textsc{Check($B_{1}$)} & \\
        \textsc{Write($B_{1}'$)} & \\
        \textsc{Unlock($B_{1}$)} & \\
        \textsc{Lock-X(Order $O_{1}$)} & \\
        \textsc{Write(Order $O_{1}$)} & \\
        \textsc{unlock(Order $O_{1}$)} & \\
        \hline
        \end{tabularx}
    \end{adjustbox}
\end{center}
\vspace*{10pt}
We add locks before each read, and unlock after each write.
This ensures that no two transactions are accessing/altering the same data point at the same time,
which could lead to conflicts.

\pagebreak

\subsection*{Non-Conflict-Serializable Schedule}
\begin{center}
    \begin{adjustbox}{width=\textwidth}
        \begin{tabularx}{\textwidth}{|X|X|}
        \hline
        \multirow{2}{*}{\textbf{Transaction-1 ($T_{1}$)}} & \multirow{2}{*}{\textbf{Transaction-2 ($T_{2}$)}} \\
        & \\ \hline
        & \textsc{Lock-X($Q$)} \\
        & \textsc{Read($Q$)} \\
        & \textsc{$Q := Q - q_{2}$} \\
        & \textsc{Write($Q$)} \\
        & \textsc{Unlock($Q$)} \\
        & \textsc{Lock-X($B_{2}$)} \\
        & \textsc{Read($B_{2}$)} \\
        & \textsc{Write($B_{2}'$)} \\
        & \textsc{Unlock($B_{2}$)} \\
        \textsc{Lock-X($Q$)} & \\
        \textsc{Read($Q$)} & \\
        \textsc{$Q := Q - q_{1}$} & \\
        \textsc{Write($Q$)} & \\
        \textsc{Unlock($Q$)} & \\
        & \textsc{Lock-X(Order $O_{2}$)} \\
        & \textsc{Write(Order $O_{2}$)} \\
        & \textsc{Unlock(Order $O_{2}$)} \\
        \textsc{Lock-X($B_{1}$)} & \\
        \textsc{Read($B_{1}$)} & \\
        \textsc{Check($B_{1}$)} & \\
        \textsc{Write($B_{1}'$)} & \\
        \textsc{Unlock($B_{1}$)} & \\
        \textsc{Lock-X(Order $O_{1}$)} & \\
        \textsc{Write(Order $O_{1}$)} & \\
        \textsc{unlock(Order $O_{1}$)} & \\
        \hline
        \end{tabularx}
    \end{adjustbox}
\end{center}
\vspace*{10pt}
We add locks before each read, and unlock after each write.
This ensures that no two transactions are accessing/altering the same data point at the same time,
which could lead to conflicts.

\pagebreak

\subsection*{Non-Conflict-Serializable Schedule with Locks}
\begin{center}
    \begin{adjustbox}{width=\textwidth}
        \begin{tabularx}{\textwidth}{|X|X|}
        \hline
        \multirow{2}{*}{\textbf{Transaction-1 ($T_{1}$)}} & \multirow{2}{*}{\textbf{Transaction-2 ($T_{2}$)}} \\
        & \\ \hline
        & \textsc{Lock-X($Q$)} \\
        & \textsc{Read($Q$)} \\
        & \textsc{$Q := Q - q_{2}$} \\
        & \textsc{Write($Q$)} \\
        & \textsc{Unlock($Q$)} \\
        & \textsc{Lock-X($B_{2}$)} \\
        & \textsc{Read($B_{2}$)} \\
        & \textsc{Write($B_{2}'$)} \\
        & \textsc{Unlock($B_{2}$)} \\
        \textsc{Lock-X($Q$)} & \\
        \textsc{Read($Q$)} & \\
        \textsc{$Q := Q - q_{1}$} & \\
        \textsc{Write($Q$)} & \\
        \textsc{Unlock($Q$)} & \\
        & \textsc{Lock-X(Order $O_{2}$)} \\
        & \textsc{Write(Order $O_{2}$)} \\
        & \textsc{Unlock(Order $O_{2}$)} \\
        \textsc{Lock-X($B_{1}$)} & \\
        \textsc{Read($B_{1}$)} & \\
        \textsc{Check($B_{1}$)} & \\
        \textsc{Write($B_{1}'$)} & \\
        \textsc{Unlock($B_{1}$)} & \\
        \textsc{Lock-X(Order $O_{1}$)} & \\
        \textsc{Write(Order $O_{1}$)} & \\
        \textsc{unlock(Order $O_{1}$)} & \\
        \hline
        \end{tabularx}
    \end{adjustbox}
\end{center}
\vspace*{10pt}
We add locks before each read, and unlock after each write.
This ensures that no two transactions are accessing/altering the same data point at the same time,
which could lead to conflicts.

\pagebreak

\subsection*{Transaction Pair 2}
Consider the case where a supplier adds products to stock (i.e., increases the quantity), and
simultaneously, a customer buys the same product (i.e. decreases the quantity). \\
Here, $Q$ represents the quantity (in stock) of Product 1. $B_{2}$ represents the balance of the customer in
$T_{2}$, while $O_{2}$ represents their order.

\begin{center}
    \begin{adjustbox}{width=\textwidth}
        \begin{tabularx}{\textwidth}{|X|X|}
        \hline
        \multirow{2}{*}{\textbf{Transaction-1 ($T_{1}$)}} & \multirow{2}{*}{\textbf{Transaction-2 ($T_{2}$)}} \\
        & \\ \hline
        \textsc{Read($Q$)} & \textsc{Read($Q$)} \\
        \textsc{$Q := Q + q_{1}$} & \textsc{$Q := Q - q_{2}$} \\
        \textsc{Write($Q$)} & \textsc{Write($Q$)} \\
        \textsc{Commit} & \textsc{Read($B_{2}$)} \\
        & \textsc{Write($B_{2}'$)} \\
        & \textsc{Write(Order $O_{2}$)} \\
        & \textsc{Commit} \\
        \hline
        \end{tabularx}
    \end{adjustbox}
\end{center}
\vspace*{5pt}

\subsection*{Conflict-Serializable Schedule}
\begin{center}
    \begin{adjustbox}{width=\textwidth}
        \begin{tabularx}{\textwidth}{|X|X|}
        \hline
        \multirow{2}{*}{\textbf{Transaction-1 ($T_{1}$)}} & \multirow{2}{*}{\textbf{Transaction-2 ($T_{2}$)}} \\
        & \\ \hline
        & \textsc{Read($Q_{1}$)} \\
        & \textsc{$Q := Q - q_{2}$} \\
        & \textsc{Write($Q$)} \\
        \textsc{Read($Q$)} & \\
        \textsc{$Q := Q + q_{1}$} & \\
        \textsc{Write($Q$)} & \\
        & \textsc{Read($B_{2}$)} \\
        & \textsc{Check($B_{2}$)} \\
        & \textsc{Write($B_{2}'$)} \\
        & \textsc{Write(Order $O_{2}$)} \\
        \hline
        \end{tabularx}
    \end{adjustbox}
\end{center}
\vspace*{10pt}
This schedule is conflict serializable as we can repeatedly move the three statements from $T_{1}$ upwards, such that
both the transactions get executed serially ($T_{2}$ after $T_{1}$).

\pagebreak

\subsection*{Conflict-Serializable Schedule with Locks}
\begin{center}
    \begin{adjustbox}{width=\textwidth}
        \begin{tabularx}{\textwidth}{|X|X|}
        \hline
        \multirow{2}{*}{\textbf{Transaction-1 ($T_{1}$)}} & \multirow{2}{*}{\textbf{Transaction-2 ($T_{2}$)}} \\
        & \\ \hline
        & \textsc{Read($Q_{1}$)} \\
        & \textsc{$Q := Q - q_{2}$} \\
        & \textsc{Write($Q$)} \\
        \textsc{Read($Q$)} & \\
        \textsc{$Q := Q + q_{1}$} & \\
        \textsc{Write($Q$)} & \\
        & \textsc{Read($B_{2}$)} \\
        & \textsc{Check($B_{2}$)} \\
        & \textsc{Write($B_{2}'$)} \\
        & \textsc{Write(Order $O_{2}$)} \\
        \hline
        \end{tabularx}
    \end{adjustbox}
\end{center}
\vspace*{10pt}
This schedule is conflict serializable as we can repeatedly move the three statements from $T_{1}$ upwards, such that
both the transactions get executed serially ($T_{2}$ after $T_{1}$).

\subsection*{Non-Conflict-Serializable Schedule}
\begin{center}
    \begin{adjustbox}{width=\textwidth}
        \begin{tabularx}{\textwidth}{|X|X|}
        \hline
        \multirow{2}{*}{\textbf{Transaction-1 ($T_{1}$)}} & \multirow{2}{*}{\textbf{Transaction-2 ($T_{2}$)}} \\
        & \\ \hline
        & \textsc{Read($Q_{1}$)} \\
        & \textsc{$Q := Q - q_{2}$} \\
        & \textsc{Write($Q$)} \\
        \textsc{Read($Q$)} & \\
        \textsc{$Q := Q + q_{1}$} & \\
        \textsc{Write($Q$)} & \\
        & \textsc{Read($B_{2}$)} \\
        & \textsc{Check($B_{2}$)} \\
        & \textsc{Write($B_{2}'$)} \\
        & \textsc{Write(Order $O_{2}$)} \\
        \hline
        \end{tabularx}
    \end{adjustbox}
\end{center}
\vspace*{10pt}
This schedule is conflict serializable as we can repeatedly move the three statements from $T_{1}$ upwards, such that
both the transactions get executed serially ($T_{2}$ after $T_{1}$).

\subsection*{Non-Conflict-Serializable Schedule with Locks}
\begin{center}
    \begin{adjustbox}{width=\textwidth}
        \begin{tabularx}{\textwidth}{|X|X|}
        \hline
        \multirow{2}{*}{\textbf{Transaction-1 ($T_{1}$)}} & \multirow{2}{*}{\textbf{Transaction-2 ($T_{2}$)}} \\
        & \\ \hline
        & \textsc{Read($Q_{1}$)} \\
        & \textsc{$Q := Q - q_{2}$} \\
        & \textsc{Write($Q$)} \\
        \textsc{Read($Q$)} & \\
        \textsc{$Q := Q + q_{1}$} & \\
        \textsc{Write($Q$)} & \\
        & \textsc{Read($B_{2}$)} \\
        & \textsc{Check($B_{2}$)} \\
        & \textsc{Write($B_{2}'$)} \\
        & \textsc{Write(Order $O_{2}$)} \\
        \hline
        \end{tabularx}
    \end{adjustbox}
\end{center}
\vspace*{10pt}
This schedule is conflict serializable as we can repeatedly move the three statements from $T_{1}$ upwards, such that
both the transactions get executed serially ($T_{2}$ after $T_{1}$).

\vfill \pagebreak

    \hspace{0pt}
    \vfill
    \begin{center}
        \Huge \textbf{User Guide} \\
        \vspace*{5pt}
        \Large \textit{How to use ElectroBase Management System}
    \end{center}
    \vfill
    \pagebreak

    \section*{\Huge Frontend Development}
    \vspace*{10pt}

    The frontend development of the application has been completed within the timeframe of Deadline-6, the final deadline for the project.
    The website is currently only hosted on a developmental server, while the database is hosted on a local machine.

    \section*{\Huge Navigation through the application}
    \vspace*{10pt}

    The frontend of the application is designed to be as intuitive as possible.
    Although there remain some minor bugs and unimplemented features, the application is fully functional and can be used by all stakeholders.
    The UI is very simple and easy to understand, and any user can navigate through the application with ease.

    \subsection*{Usage as Customers}
    \subsubsection*{Without creating an account}
    The application can be used without creating an account, but the user will not be able to access the full functionality of the application.
    Without creating an account, the user can view limited data, such as the top-3 selling products, and the top-3 selling suppliers.
    They can also browse through the product calalogue.

    \subsubsection*{Creating an account}
    To create an account, the user can follow one of two steps:
    \begin{enumerate}
        \item
        Click on the \texttt{Log In} button located at the top right of the navigation bar.
        From there, the user can click on the \texttt{Create an account} button to create an account.
        \item
        While browsing the product catalogue, if the user clicks on the \texttt{Add to Cart} button, they will be prompted to log in first.
        From there, they can create an account.
    \end{enumerate}

    \subsubsection*{Logging in}
    After logging in, the user can add products to their cart.
    They can use the \texttt{Cart} button on the navigation bar to proceed to the checkout page.
    They can review their order before placing it, and if confirmed, they are redirected to the payment gateway and the order will be placed.
    \vspace*{10pt} \\
    On the profile page, the customer can view their personal information\footnote{
        The feature to update personal information is not yet implemented.
    }, their order history, and the current active (undelivered) orders.

    \subsection*{Usage as Admins}
    \subsubsection*{Logging In}
    The admin\footnote{
        Currently, there are only two admin accounts, both having the same functionality.
    } can log in to the admin panel by clicking on the \texttt{Log In as Admin} button shown in the footer.
    This button is displayed on almost all pages in the footer, even on the user-login page.

    \subsubsection*{How to use}
    Once logged in, the admin can use the admin panel to perform various tasks\footnote{
        The feature to add records through the admin panel is not yet implemented.
    }, such as viewing the stocks through the catalogue, viewing orders, order statistics,
    and viewing the sales statistics.

    \subsection*{Usage as Suppliers and Delivery Agents}
    Both suppliers and delivery agents have similar interfaces to users for logging in.
    They can log in by clicking on the \texttt{Log In} button located at the top right of the navigation bar.
    From there, they can perform the tasks specific to their role.

    \vfill \pagebreak

    \section*{References and Bibliography}

    \begin{itemize}
        \item \href{https://mdbootstrap.com/docs/b4/jquery/tables/sort/}{\color{blue}\underline{Sortable tables in HTML}} \\
        The JavaScript code to implement sorting on tables in HTML is taken from this website.
        \item \href{https://www.w3schools.com/howto/howto_css_animate_buttons.asp}{\color{blue}\underline{Button animations using CSS}}
        \item \href{https://codepen.io/hesguru/pen/BaybqXv}{\color{blue}\underline{Star-Rating implementation using CSS}}
        \item \href{https://www.geeksforgeeks.org/types-of-schedules-in-dbms/}{\color{blue}\underline{Types of schedules in DBMS}}
    \end{itemize}
\end{document}